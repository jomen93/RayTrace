
\section{Kerr Spacetime}.

The Kerr spacetime in Boyer-Lindquist coordinates is given by the line element
\begin{align}
	ds^2 = &-\left( 1- \frac{2Mr}{\Sigma} \right) dt^2 - -\frac{4Mar\sin^2 \theta}{\Sigma} dt d\phi\\
	&+ \frac{\Sigma}{\Delta} dr^2 +\Sigma d\theta^2 + \sin^2 \theta \left( r^2 + a^2 +\frac{2Ma^2 r \sin^2 \theta}{\Sigma} \right) d\phi^2,
\end{align}

where

\begin{align}
	\Sigma &= r^2 + a^2 \cos \theta \\
	\Delta &= r^2 - 2Mr + a^2.
\end{align}

Therefore, the potentials in the description of a particle moving in this spacetime reduce to
\begin{align}
	R &= W^2 - (r^2 - 2Mr + a^2) \left[ r^2 + (L-aE)^2 + Q \right]\\
	W &= E(r^2 + a^2) - aL\\
	\Theta &= Q - \cos^2 \theta \left[ a^2 (1-E^2) + \frac{L^2}{\sin^2 \theta} \right]
\end{align}

Then, we have the function
\begin{align}
	\Xi = &R + \Delta \Theta \nonumber \\
	\Xi = &W^2 - \Delta  \left[ r^2 + (L-aE)^2 + Q \right] + \Delta Q  - \Delta \cos^2 \theta  \left[ a^2 (1-E^2) + \frac{L^2}{\sin^2 \theta} \right] \nonumber \\ 
	\Xi = &W^2  - \Delta \left[ r^2 + (L-aE)^2  
	+  a^2 (1-E^2)  \cos^2 \theta + \frac{\cos^2 \theta }{\sin^2 \theta} L^2  \right] 
\end{align}


In order to write the equations of motion we need the derivatives
\begin{align}
\frac{\partial \Xi}{\partial E} &= 2W(r^2 + a^2) + 2a \Delta (L-aE) + 2E a^2 \Delta \cos^2 \theta \nonumber  \\
										&= 2W(r^2 + a^2) + 2a \Delta \left( L-aE  + aE  \cos^2 \theta \right) \nonumber \\
										&= 2W(r^2 + a^2) + 2a \Delta \left( L-aE \sin^2 \theta \right)
\end{align}
\begin{align}
\frac{\partial \Xi}{\partial L} &= - 2a W - 2 \Delta (L-aE) - 2L \Delta \frac{\cos^2 \theta }{\sin^2 \theta}   \nonumber \\
						&= - 2a W + 2 aE \Delta   - 2L \Delta \left[ 1 + \frac{\cos^2 \theta }{\sin^2 \theta} \right] \nonumber \\											
						&= - 2a W + 2 aE \Delta   - 2L \Delta \csc^2 \theta . 
\end{align}

Since 
\begin{equation}
\frac{\partial \Delta}{\partial r} = 2(r-M)
\end{equation}
and 
\begin{equation}
\frac{\partial \Sigma}{\partial r} = 2r,
\end{equation}
we obtain the derivatives
\begin{align}
\frac{\partial}{\partial r}\left( \frac{\Delta}{2\Sigma}\right) &= \frac{1}{2\Sigma} \frac{\partial \Delta}{\partial r} - \frac{\Delta}{2\Sigma^2} \frac{\partial \Sigma}{\partial r} = \frac{r-M}{\Sigma} - \frac{r\Delta}{\Sigma^2} \\
\frac{\partial}{\partial r}\left( \frac{1}{2\Sigma}\right) &= - \frac{1}{2\Sigma^2} \frac{\partial \Sigma}{\partial r} = - \frac{r}{\Sigma^2} \\
\frac{\partial}{\partial r}\left( \frac{\Xi}{2\Delta \Sigma}\right) &= \frac{1}{2\Delta \Sigma} \frac{\partial \Xi}{\partial r} -\frac{\Xi}{2\Delta^2 \Sigma} \frac{\partial \Delta}{\partial r} -\frac{\Xi}{2\Delta \Sigma^2} \frac{\partial \Sigma}{\partial r}  \nonumber \\
&= \frac{1}{2\Delta \Sigma} \frac{\partial \Xi}{\partial r} -\frac{\Xi (r-M)}{\Delta^2 \Sigma} -\frac{\Xi r}{\Delta \Sigma^2} 
\end{align}
where
\begin{equation}
\frac{\partial \Xi}{\partial r}  = 4rEW - 2(r-M) \left[ r^2 + (L-aE)^2  
	+  a^2 (1-E^2)  \cos^2 \theta + \frac{\cos^2 \theta }{\sin^2 \theta} L^2  \right] -2r \Delta.
\end{equation}

Similarly, since
\begin{equation}
\frac{\partial \Delta}{\partial \theta} = 0
\end{equation}
and 
\begin{equation}
\frac{\partial \Sigma}{\partial \theta} = -2a^2 \cos \theta \sin \theta ,
\end{equation}
we have the derivatives
\begin{align}
\frac{\partial}{\partial \theta}\left( \frac{\Delta}{2\Sigma}\right) &= - \frac{\Delta}{2\Sigma^2} \frac{\partial \Sigma}{\partial \theta} =  \frac{\Delta}{\Sigma^2} a^2 \cos \theta \sin \theta \\
\frac{\partial}{\partial \theta}\left( \frac{1}{2\Sigma}\right) &= - \frac{1}{2\Sigma^2} \frac{\partial \Sigma}{\partial \theta} =  \frac{a^2 \cos \theta \sin \theta}{\Sigma^2}  \\
\frac{\partial}{\partial \theta}\left( \frac{\Xi}{2\Delta \Sigma}\right) &= \frac{1}{2\Delta\Sigma} \frac{\partial \Xi}{\partial \theta} - \frac{\Xi}{2\Delta\Sigma^2} \frac{\partial \Sigma}{\partial \theta} \nonumber \\
&= \frac{1}{2\Delta\Sigma} \left[ 2\Delta a^2 (1-E^2) \cos \theta \sin \theta + 2\Delta L^2 \cot \theta \csc ^2 \theta \right] + \frac{\Xi}{\Delta\Sigma^2} a^2 \cos \theta \sin \theta \nonumber \\
&= \frac{1}{\Sigma} \left[  a^2 (1-E^2) \cos \theta \sin \theta + L^2 \cot \theta \csc ^2 \theta \right] + \frac{\Xi}{\Delta\Sigma^2} a^2 \cos \theta \sin \theta 
\end{align}

Using this expressions, the equations of motion of a particle in this spacetime are given by the Hamilton's equations

The equations of motion of a particle in this spacetime are 

\begin{align*}
	\dot{t} &= \frac{1}{2\Delta \Sigma} \frac{\partial \Xi}{\partial E}\\
	\dot{r} &= \frac{\Delta}{\Sigma} p_r \\
	\dot{\theta} &= \frac{p_\theta}{\Sigma}\\
	\dot{\phi} &= - \frac{1}{2\Delta \Sigma} \frac{\partial \Xi}{\partial L}\\	
\end{align*}

\begin{align*}
\dot{p}_t &= 0\\
\dot{p}_r &= -\frac{\partial}{\partial r}\left( \frac{\Delta}{2\Sigma}\right) p_r^2 - \frac{\partial}{\partial r}\left( \frac{1}{2\Sigma}\right) p_\theta^2 + \frac{\partial}{\partial r}\left( \frac{\Xi}{2\Delta \Sigma}\right) \\
\dot{p}_\theta &= -\frac{\partial}{\partial \theta}\left( \frac{\Delta}{2\Sigma}\right) p_r^2 - \frac{\partial}{\partial \theta}\left( \frac{1}{2\Sigma}\right) p_\theta^2 + \frac{\partial}{\partial \theta}\left( \frac{\Xi}{2\Delta \Sigma}\right)\\
\dot{p}_\phi &= 0\\	
\end{align*}



\begin{align*}
	\dot{t} &= \frac{1}{2r^2(r^2 -2Mr)} 2E r^4 = \frac{E r^2}{(r^2 -2Mr)}  \\
	\dot{r} &= p_r \left(1 - \frac{2M}{r} \right)  \\
	\dot{\theta} &= \frac{p_\theta}{r^2}\\
	\dot{\phi} &= - \frac{1}{2r^2(r^2 -2Mr)} \left[ - 2(r^2 -2Mr)L \csc^2 \theta \right] = \frac{L}{r^2 \sin^2 \theta}\\	
\end{align*}

\begin{align*}
\dot{p}_t &= 0\\
\dot{p}_r &= - \frac{M}{r^2}p_r^2 +  \frac{p_\theta^2}{r^3} -\frac{E^2 M}{(r-2M)^2}  + \frac{L^2}{r^3 \sin^2 \theta} \\
\dot{p}_\theta &= \frac{\cos \theta}{\sin^3 \theta} \frac{L^2}{r^2}\\
\dot{p}_\phi &= 0\\	
\end{align*}

