
\chapter{The Minkowski Spacetime}.
The Minkowski spacetime is given by the line element
\begin{equation}
	ds^2 = -dt^2 + dx^2 + dy^2 + dz^2
\end{equation}
or in spherical coordinates
\begin{equation}
 	ds^2 = -dt^2 + dr^2 + r^2 d\theta^2 + r^2 \sin^2 \theta d\phi^2.
\end{equation} 

When compared with the standard form of the Kerr's line element in Boyer-Lindquist coordinates,
\begin{align}
	ds^2 = &-\left( 1- \frac{2Mr}{\Sigma} \right) dt^2 -\frac{4Mar\sin^2 \theta}{\Sigma} dt d\phi \nonumber  \\
	&+ \frac{\Sigma}{\Delta} dr^2 +\Sigma d\theta^2 + \sin^2 \theta \left( r^2 + a^2 +\frac{2Ma^2 r \sin^2 \theta}{\Sigma} \right) d\phi^2,
\end{align}
Minkowski's metric is obtained by taking $a=0$, $M=0$ and
\begin{align}
	\Sigma &= r^2\\
	\Delta &= r^2.
\end{align}

Therefore, the potentials in the description of a particle moving in this spacetime reduce to
\begin{align}
	R &= E^2 r^4 - r^2 \left[ r^2 + L^2 + Q \right] \\
	\Theta &= Q -  \frac{\cos^2 \theta}{\sin^2 \theta}L^2
\end{align}

Then, we have the function
\begin{align}
	\Xi &= E^2 r^4 - r^2 \left[ r^2 + L^2 + Q \right] + r^2 \left[ Q -  \frac{\cos^2 \theta}{\sin^2 \theta}L^2 \right] \nonumber \\
	\Xi &= E^2 r^4 -  r^4 -  L^2 r^2  - r^2 \frac{\cos^2 \theta}{\sin^2 \theta}L^2 
\end{align}

In order to write the equations of motion we need the derivatives
\begin{align}
\frac{\partial \Xi}{\partial E} &= 2E r^4 \\
\frac{\partial \Xi}{\partial L} &= - 2 L r^2 -2 r^2 \frac{\cos^2 \theta}{\sin^2 \theta}L
\end{align}
and also
\begin{align}
\frac{\partial}{\partial r}\left( \frac{\Delta}{2\Sigma}\right) &= \frac{\partial}{\partial r}\left( \frac{1}{2}\right) = 0 \\
\frac{\partial}{\partial r}\left( \frac{1}{2\Sigma}\right) &= \frac{\partial}{\partial r}\left( \frac{1}{2r^2}\right) = -\frac{1}{r^3}\\
\frac{\partial}{\partial r}\left( \frac{\Xi}{2\Delta \Sigma}\right) &= \frac{\partial}{\partial r}\left( \frac{E^2 r^4 -  r^4 -  L^2 r^2  - r^2 \frac{\cos^2 \theta}{\sin^2 \theta} L^2}{2r^4}\right) \nonumber \\
&= \frac{1}{2} \frac{\partial}{\partial r}\left( E^2 - 1 - \frac{L^2}{r^2} - \frac{\cos^2 \theta}{\sin^2 \theta} \frac{L^2}{r^2} \right) \nonumber \\
&= \frac{L^2}{r^3} + \frac{\cos^2 \theta}{\sin^2 \theta} \frac{L^2}{r^3} 
\end{align}

\begin{align}
\frac{\partial}{\partial \theta}\left( \frac{\Delta}{2\Sigma}\right) &= \frac{\partial}{\partial r}\left( \frac{1}{2}\right) = 0 \\
\frac{\partial}{\partial \theta}\left( \frac{1}{2\Sigma}\right) &= \frac{\partial}{\partial r}\left( \frac{1}{2r^2}\right) = 0\\
\frac{\partial}{\partial \theta}\left( \frac{\Xi}{2\Delta \Sigma}\right) &= \frac{\partial}{\partial \theta}\left( \frac{E^2 r^4 -  r^4 -  L^2 r^2  - r^2 \frac{\cos^2 \theta}{\sin^2 \theta} L^2}{2r^4}\right) \nonumber \\
&= \frac{1}{2} \frac{\partial}{\partial \theta}\left( E^2 - 1 - \frac{L^2}{r^2} - \frac{\cos^2 \theta}{\sin^2 \theta} \frac{L^2}{r^2} \right) \nonumber \\
&= \cot \theta \csc^2 \theta \frac{L^2}{r^2} 
\end{align}

Using this expressions, the equations of motion of a particle in this spacetime are given by the Hamilton's equations

\begin{align*}
	\dot{t} &= \frac{1}{2r^4} 2E r^4 = E\\
	\dot{r} &= p_r \\
	\dot{\theta} &= \frac{p_\theta}{r^2}\\
	\dot{\phi} &= - \frac{1}{2r^4} \left[ - 2 L r^2 -2 r^2 \frac{\cos^2 \theta}{\sin^2 \theta}L \right] = \left[ 1 + \frac{\cos^2 \theta}{\sin^2 \theta} \right] \frac{L}{r^2} = \frac{L}{r^2 \sin^2 \theta}\\	
\end{align*}


\begin{align*}
\dot{p}_t &= 0\\
\dot{p}_r &= \frac{p_\theta^2}{r^3} + \frac{L^2}{r^3} + \frac{\cos^2 \theta}{\sin^2 \theta} \frac{L^2}{r^3} 
          = \frac{p_\theta^2}{r^3} + \frac{1}{\sin^2 \theta} \frac{L^2}{r^3}\\
\dot{p}_\theta &= \frac{\cos \theta}{\sin^3 \theta} \frac{L^2}{r^2}\\
\dot{p}_\phi &= 0\\	
\end{align*}
